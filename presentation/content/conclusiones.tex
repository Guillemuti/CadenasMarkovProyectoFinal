\section{Conclusiones}

\begin{frame}{Conclusiones Principales}
    \begin{enumerate}
        \item<1-> \textbf{Utilidad del Estimador de McDonald}
        \begin{itemize}
            \item Permite verificar empíricamente la propiedad de $\beta$-mezcla
            \item Distingue efectivamente entre procesos con memoria larga y corta
            \item Convergencia verificada conforme $n$ crece
        \end{itemize}

        \item<2-> \textbf{S\&P 500: Memoria Larga}
        \begin{itemize}
            \item $\hat{\beta}(a)$ NO decrece hacia cero
            \item Evidencia de volatility clustering y dependencia persistente
            \item Consistente con modelos GARCH de alta persistencia
        \end{itemize}

        \item<3-> \textbf{Retrasos de Vuelos: Memoria Corta}
        \begin{itemize}
            \item $\hat{\beta}(a) \to 0$ exponencialmente rápido
            \item Sistema se recupera rápidamente de disrupciones
            \item Teoremas límite clásicos son aplicables
        \end{itemize}
    \end{enumerate}
\end{frame}

\begin{frame}{Implicaciones Teóricas y Prácticas}
    \begin{block}{Implicaciones Teóricas}
        \begin{itemize}
            \item<1-> Validación empírica del Teorema de McDonald (2015)
            \item<2-> Condiciones de convergencia verificadas en ambos casos
            \item<3-> Conexión entre ergodicidad geométrica y $\beta$-mezcla confirmada
        \end{itemize}
    \end{block}

    \vspace{0.5em}

    \begin{block}<4->{Implicaciones Prácticas}
        \begin{itemize}
            \item Modelado financiero: Requiere métodos más sofisticados que CLT clásico
            \item Sistemas operacionales (aeropuertos): Análisis estadístico estándar es válido
            \item Herramienta diagnóstica para determinar validez de supuestos estadísticos
        \end{itemize}
    \end{block}
\end{frame}

\begin{frame}{Hallazgos Clave}
    \begin{alertblock}{Contraste Fundamental}
        Los dos procesos estudiados exhiben \textbf{estructuras de dependencia cualitativamente distintas}:
    \end{alertblock}

    \vspace{0.5em}

    \begin{columns}
        \begin{column}{0.5\textwidth}
            \textbf{Mercados Financieros:}
            \begin{itemize}
                \item Dependencia persistente
                \item Memoria larga
                \item $\beta(a) \not\to 0$
                \item Requieren teoría asintótica avanzada
            \end{itemize}
        \end{column}

        \begin{column}{0.5\textwidth}
            \textbf{Sistemas Operacionales:}
            \begin{itemize}
                \item Independencia asintótica
                \item Memoria corta
                \item $\beta(a) \to 0$
                \item Teoremas clásicos aplicables
            \end{itemize}
        \end{column}
    \end{columns}
\end{frame}

\begin{frame}{Trabajo Futuro}
    \begin{block}{Extensiones Metodológicas}
        \begin{itemize}
            \item Explorar otros coeficientes de mezcla ($\alpha$, $\rho$, $\phi$)
            \item Desarrollar tests formales de hipótesis para $\beta$-mezcla
            \item Estudiar tasas de convergencia del estimador
        \end{itemize}
    \end{block}

    \vspace{0.5em}

    \begin{block}{Aplicaciones Adicionales}
        \begin{itemize}
            \item Cadenas de Markov en biología (modelos de población)
            \item Procesos climáticos y meteorológicos
            \item Redes neuronales recurrentes y series temporales
            \item Análisis de tráfico de redes y telecomunicaciones
        \end{itemize}
    \end{block}
\end{frame}

\begin{frame}{Referencias Principales}
    \begin{thebibliography}{9}
        \bibitem{mcdonald2015}
        McDonald, D. J. (2015). Histogram estimation from dependent data. \textit{arXiv preprint arXiv:1508.07322}.

        \bibitem{bradley2005}
        Bradley, R. C. (2005). Basic properties of strong mixing conditions. \textit{Probability Surveys}, 2, 107-144.

        \bibitem{douc2018}
        Douc, R., Moulines, E., Priouret, P., \& Soulier, P. (2018). \textit{Markov chains}. Springer.

        \bibitem{tierney1994}
        Tierney, L. (1994). Markov chains for exploring posterior distributions. \textit{The Annals of Statistics}, 22(4), 1701-1728.
    \end{thebibliography}
\end{frame}
